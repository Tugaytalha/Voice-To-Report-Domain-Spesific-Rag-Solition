\chapter{Introduction}
\label{cha:introduction}

\section{Background and Motivation}

In modern clinical practice, medical imaging is a cornerstone of diagnosis, treatment planning, and patient monitoring. Radiologists, the physicians who interpret these images (such as X-rays, CT scans, and MRIs), convey their expert findings through a dictated report. This dictation, an unstructured stream of consciousness containing complex medical terminology, must then be transcribed and formatted into a structured, coherent, and clinically precise document. This final report becomes a critical part of the patient's electronic health record (EHR), serving as a legal document and a primary communication tool between the radiologist and other healthcare providers.

The traditional workflow for report generation is fraught with inefficiencies. It is a manually intensive process that consumes a significant portion of a radiologist's valuable time, diverting their focus from image interpretation. The transcription process, whether performed by the radiologist or a transcriptionist, is susceptible to errors, omissions, and inconsistencies. This can lead to delays in patient care, miscommunication between clinicians, and a decrease in the overall quality of care. Furthermore, the lack of standardization in unstructured reports makes it exceedingly difficult to extract data for large-scale research, quality assurance, or the development of clinical decision support systems.

This project is motivated by the pressing need to address these challenges. The recent advancements in Artificial Intelligence, specifically in Natural Language Processing (NLP), Speech-to-Text (STT), and Large Language Models (LLMs), present an unprecedented opportunity to automate and revolutionize the radiology reporting process. By creating a system that can listen to a radiologist's dictation and automatically generate a high-quality, structured report, we can significantly reduce administrative burden, improve report accuracy and consistency, and unlock the valuable data trapped within these clinical documents.

\section{Project Objectives}

The primary goal of this project, "InsightBridge," is to design, implement, and evaluate a voice-driven system for the automated generation of structured radiology reports. The system aims to transform the free-form, spoken dictation of a radiologist into a complete, clinically accurate report ready for final review.

The specific objectives are as follows:
\begin{enumerate}
    \item \textbf{High-Fidelity Medical Speech-to-Text:} To implement and optimize an STT module capable of accurately transcribing radiological dictations, with a special focus on handling complex medical terminology and preserving nuance.
    \item \textbf{Develop a Specialized RAG for Report Generation:} To build a Retrieval-Augmented Generation pipeline tailored for this task. This involves creating a knowledge base of radiological templates and standard phrases, and a retriever that can select the most relevant information based on the content of the dictation.
    \item \textbf{Automated Report Structuring:} To use an LLM to analyze the transcribed dictation, extract key findings (e.g., measurements, anatomical locations, abnormalities), and intelligently populate the sections of a standard radiology report template (e.g., Technique, Findings, Impression).
    \item \textbf{Intuitive Clinical User Interface:} To create a user-friendly interface that seamlessly integrates into the radiologist's workflow. The UI must allow for easy dictation, rapid review of the generated report, and efficient editing capabilities.
    \item \textbf{Domain-Specific Adaptation:} To ensure the system is adaptable to different radiological sub-specialties and institutional reporting standards by designing a modular and configurable architecture.
\end{enumerate}

\section{Report Structure}

This report provides a comprehensive overview of the InsightBridge project, detailing its clinical context, technical implementation, and evaluation.
\begin{itemize}
    \item \textbf{Chapter 2 (System Architecture and Design):} Presents the high-level architecture of the voice-to-report pipeline, explaining the role of each component from voice capture to final report generation.
    \item \textbf{Chapter 3 (Implementation Details):} Provides a technical deep-dive into the core scripts, algorithms, and libraries used, with a focus on how they were adapted for the radiology domain.
    \item \textbf{Chapter 4 (Experiments and Evaluation):} Describes the methodology for evaluating the system, including the domain-specific datasets used and metrics designed to assess clinical and structural accuracy.
    \item \textbf{Chapter 5 (Conclusion and Future Work):} Summarizes the project's contributions to the field of medical informatics, discusses its limitations, and outlines potential paths for future enhancement and clinical integration.
\end{itemize}